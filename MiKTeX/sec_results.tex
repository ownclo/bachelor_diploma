\section{Результаты сравнительного анализа моделей}
\label{sec:results}

\subsection{Сравнение гистограмм}
\label{sub:histresults}

Типовая гистограмма представлена на рисунке~\ref{fig:romanCapShisto}.
Дискретная авторегрессионная модель с отрицательным биномиальным
распределением в качестве остаточного (см. подраздел~\ref{sse:darp})
хорошо отражает моделируемый тип трафика. Марковские модели
в силу ограниченного количества квантов (здесь и далее представлены
реализации марковских моделей с двадцатью состояниями)
характеризуются импульсным характером функции распределения.
Нормированность гистограммы и провалы в отрезках, не содержащих
аппроксимирующих значений, приводят к увеличению значений
оценки функции плотности вероятностей относительно гистограммы
исходной последовательности.

Гистограммы на рисунке~\ref{fig:romanCapShisto} не демонстрируют
очевидного преимущества использования неравномерного квантования
(подраздел~\ref{sse:markkmeans}) перед равномерным (подраздел~\ref{sse:marksimple}).
Однако в случае распределения с ``длинным хвостом'' ситуация в корне меняется.
На рисунке~\ref{fig:borisFullMKhisto} представлено сравнение гистограмм
для тестового видео с ``длинным хвостом'' распределения
и для марковской модели с нелинейным квантованием (MK).
Видно, что почти все аппроксимирующие
значения находятся в области ``купола'' распределения, или
в области концентрации энергии последовательности.
На следующем
наборе гистограмм (рисунок~\ref{fig:borisFullMLhisto})
для той же видеопоследовательности дополнительно приведена
гистограмма для простой цепи Маркова с равномерным квантованием.
Так как в рамках этой модели область значений размеров кадров
разбивается на равные отрезки, на область ``купола'' пришлось
всего три состояния, а для набора статистики переходов в область
``хвоста'' просто не хватило данных.

В целом, сравнение гистограмм распределений размеров кадров
показывает, что нелинейная марковская модель лучше отражает
распределение размеров кадров тестовых последовательностей.
В то же время, аппроксимация распределения обратным биномиальным
распределением тоже достаточно точна.

\subsection{Сравнение функций автокорреляции}

Сравнение функций автокорреляции (англ. ACF) целесообразно осуществлять
отдельно при маленьких при больших значениях сдвига
для анализа краткосрочной и долгосрочной зависимостей.

Сравнение ACF для краткосрочных зависимостей на рисунке~\ref{fig:akiyosrd}
показывает, что для тестового видео Akiyo предложенная модель с
нелинейным квантованием лучше других соответствует исходной
последовательности. В целом, все предложенные модели достаточно
точно отражают краткосрочные зависимости в трафике видеоконференций.

Анализ долгосрочных зависимостей (рисунок~\ref{fig:borisFullacflong})
показывает, что в некоторых случаях предложенные модели
успешно отражают и долгосрочные зависимости. Однако
для тестовых последовательностей высокого качества в целом
характерна неспособность отразить субэкспоненциальное
убывание автокорреляции, продемонстрированное на рисунке~\ref{fig:romanCpamMacf}.

\subsection{Сравнение характеристик качества обслуживания}

Анализ характеристик качества обслуживания может давать
дополнительные сведения о применимости той или иной
модели для данного тестового видео. Так, на рисунке~\ref{fig:borisFullbucket}
приведено сравнение коэффикиента потери при переполнении буфера
в рамках теста с текущим ведром (описан в подразделе~\ref{sse:bucket}).
Тест позволяет сделать вывод об адекватности использования
марковской нелинейной модели для этого видео. При этом
марковская линейная модель (ML) демонстрирует худший результат
на этом тесте.

Анализ задержки в некоторых случаях может давать ценную информацию
о качестве модели. Так, в случае с тестовым видео Akiyo (рисунок~\ref{fig:akiyodelaypresentation})
лучше всего по критерию задержки исходному трейсу соответствует дискретная
авторегрессионная модель второго порядка. Марковская нелинейная модель и на этом
тесте превосходит линейную. В некоторых случаях анализ задержки выявляет
неспособность используемых моделей отражать тестовые видео, в особенности
видео высокого качества с ``длинным хвостом''. Представленная на рисунке~\ref{fig:borisFulldelay}
ситуация является следствием серии из больших значений в моделируемом трейсе,
в результате которой буфер отправителя единовременно заполняется большим объёмом данных,
что вызывает лавинообразный рост задержки при передаче последующих кадров и сказывается на среднем
значении этой величины при любой пропускной способности канала.

Сравнение среднего джиттера для разных моделей (рисунок~\ref{fig:akiyojitter})
в некоторых случаях позволяет признать неадекватными некоторые модели.
Так, в случае с последовательностью Akiyo исследование джиттера побуждает
отказаться от моделирования этого видео с помощью модели DAR(1) и линейной
марковской модели.

\subsection{Результаты сравнения моделей}

Сравнительный анализ моделей позволяет прийти к следующим заключениям:

\begin{itemize}
    \item Выбранные модели хорошо отражают краткосрочные
        зависимости в моделируемом трафике
    \item Предложенная марковская модель с нелинейным квантованием
        превосходит линейную модель для б\'{о}льшей части
        тестовых видеопоследовательностей
    \item Предлложенная оценка задержки и джиттера является
        полезным дополнением к имеющимся методикам сравнительного анализа моделей
    \item Тестовые видео хорошего качества обладают
        долгосрочными зависимостями
    \item Предложенные модели ``работают'' для тестовых
        видео плохого качества, но не способны отражать
        долгосрочные зависимости качественных видеопоследовательностей
\end{itemize}

    \begin{figure}[h]
        \begin{center}
            \includegraphics[width=\textwidth]{romanCapS_histo.pdf}
        \end{center}
        \caption{Сравнение гистограмм размеров кадров для различных моделей (RomanS)}
        \label{fig:romanCapShisto}
    \end{figure}

    \begin{figure}[h]
        \begin{center}
            \includegraphics[width=\textwidth]{borisFullMK_histo.pdf}
        \end{center}
        \caption{Сравнение гистограмм размеров кадров для различных моделей (Boris)}
        \label{fig:borisFullMKhisto}
    \end{figure}

    \begin{figure}[h]
        \begin{center}
            \includegraphics[width=\textwidth]{borisFullML_histo.pdf}
        \end{center}
        \caption{Сравнение гистограмм размеров кадров для различных моделей (Boris)}
        \label{fig:borisFullMLhisto}
    \end{figure}

    \begin{figure}[h]
        \begin{center}
            \includegraphics[width=\textwidth]{akiyo_srd.pdf}
        \end{center}
        \caption{Сравнение автокорреляции для различных моделей (SRD) (akiyo)}
        \label{fig:akiyosrd}
    \end{figure}

    \begin{figure}[h]
        \begin{center}
            \includegraphics[width=\textwidth]{borisFull_acflong.pdf}
        \end{center}
        \caption{Сравнение автокорреляции для различных моделей (LRD) (Boris)}
        \label{fig:borisFullacflong}
    \end{figure}

    \begin{figure}[h]
        \begin{center}
            \includegraphics[width=\textwidth]{romanCapM_acf.pdf}
        \end{center}
        \caption{Сравнение автокорреляции для различных моделей (LRD) (RomanM)}
        \label{fig:romanCpamMacf}
    \end{figure}

    \begin{figure}[h]
        \begin{center}
            \includegraphics[width=\textwidth]{borisFull_bucket.pdf}
        \end{center}
        \caption{Тест с текущим ведром (Boris)}
        \label{fig:borisFullbucket}
    \end{figure}

    \begin{figure}[h]
        \begin{center}
            \includegraphics[width=\textwidth]{akiyo_delay_presentation.pdf}
        \end{center}
        \caption{Задержка при передаче по CBR-каналу (akiyo)}
        \label{fig:akiyodelaypresentation}
    \end{figure}

    \begin{figure}[h]
        \begin{center}
            \includegraphics[width=\textwidth]{borisFull_delay.pdf}
        \end{center}
        \caption{Задержка при передаче по CBR-каналу (Boris)}
        \label{fig:borisFulldelay}
    \end{figure}

    \begin{figure}[h]
        \begin{center}
            \includegraphics[width=\textwidth]{akiyo_jitter.pdf}
        \end{center}
        \caption{Джиттер при передаче по CBR-каналу (akiyo)}
        \label{fig:akiyojitter}
    \end{figure}
