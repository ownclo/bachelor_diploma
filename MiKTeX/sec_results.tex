\section{Результаты сравнительного анализа моделей}
\label{sec:results}

    \begin{figure}[h]
        \begin{center}
            \includegraphics[width=\textwidth]{romanCapS_histo.pdf}
        \end{center}
        \caption{Сравнение гистограмм размеров кадров для различных моделей (rs)}
    \end{figure}

    \begin{figure}[h]
        \begin{center}
            \includegraphics[width=\textwidth]{borisFullMK_histo.pdf}
        \end{center}
        \caption{Сравнение гистограмм размеров кадров для различных моделей (b)}
    \end{figure}

    \begin{figure}[h]
        \begin{center}
            \includegraphics[width=\textwidth]{borisFullML_histo.pdf}
        \end{center}
        \caption{Сравнение гистограмм размеров кадров для различных моделей (b)}
    \end{figure}

    \begin{figure}[h]
        \begin{center}
            \includegraphics[width=\textwidth]{akiyo_srd.pdf}
        \end{center}
        \caption{Сравнение автокорреляции для различных моделей (SRD) (akiyo)}
    \end{figure}

    \begin{figure}[h]
        \begin{center}
            \includegraphics[width=\textwidth]{borisFull_acflong.pdf}
        \end{center}
        \caption{Сравнение автокорреляции для различных моделей (LRD) (b)}
    \end{figure}

    \begin{figure}[h]
        \begin{center}
            \includegraphics[width=\textwidth]{romanCapM_acf.pdf}
        \end{center}
        \caption{Сравнение автокорреляции для различных моделей (LRD) (rm)}
    \end{figure}

    \begin{figure}[h]
        \begin{center}
            \includegraphics[width=\textwidth]{borisFull_bucket.pdf}
        \end{center}
        \caption{Тест с текущим ведром (b)}
    \end{figure}

    \begin{figure}[h]
        \begin{center}
            \includegraphics[width=\textwidth]{akiyo_delay_presentation.pdf}
        \end{center}
        \caption{Задержка при передаче по CBR-каналу (akiyo)}
    \end{figure}

    \begin{figure}[h]
        \begin{center}
            \includegraphics[width=\textwidth]{borisFull_delay.pdf}
        \end{center}
        \caption{Задержка при передаче по CBR-каналу (b)}
    \end{figure}

    \begin{figure}[h]
        \begin{center}
            \includegraphics[width=\textwidth]{akiyo_jitter.pdf}
        \end{center}
        \caption{Джиттер при передаче по CBR-каналу (akiyo)}
    \end{figure}

    \begin{itemize}
        \item Выбранные модели хорошо отражают краткосрочные
            зависимости в моделируемом трафике
        \item Предложенная марковская модель с нелинейным квантованием
            превосходит линейную модель для б\`{о}льшей части
            тестовых видеопоследовательностей
        \item Предлложенная оценка задержки и джиттера является
            полезным дополнением к имеющимся методикам
        \item Тестовые видео хорошего качества обладают
            долгосрочными зависимостями
        \item Предложенные модели ``работают'' для тестовых
            видео плохого качества, но не способны отражать
            долгосрочные зависимости качественных видеопоследовательностей
    \end{itemize}

