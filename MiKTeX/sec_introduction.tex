\sectioncentered*{Введение}
\addcontentsline{toc}{section}{Введение}
\label{sec:intro}

Экспоненциальный рост ресурсов сети Интернет~\cite{}
сделал обыденным
потоковую передачу видеотрафика в режиме реального времени.
В настоящее время организация видеозвонков и видеоконференций
на базе интернет-сервисов, подобных Skype~\cite{} и
GoogleTalk~\cite{}, является повсеместной практикой.
Заинтересованность крупных корпораций в программном
обеспечении телеконференций и удалённых совещаний
привела к созданию многочисленных коммерческих продуктов
по передаче видео по сети Интернет, таким, как WebEx~\cite{}
и Adobe Connect~\cite{}. Подобные программные решения
существенно поднимают планку требований к характеристикам
качества обслуживания, предъявляемых к сетям передачи
данных. Приведённые соображения свидетельствуют и о
необходимости в целенаправленном развитии мультимедиа-сетевых
технологий.

Одним из основных составляющих мультимедиа-сетевых технологий
является сжатие данных, или кодирование источника~\cite{}
в терминах теории информации, задачей которого является
уменьшение избыточности мультимедиа-источника (аудио,
видео или изображения). Сжатие видео -- это процедура,
позволяющая уменьшить объём информации, необходимый
для представления цифрового видеосигнала, предшествующая передаче
этого сигнала по сети или сохранения на электронном носителе~\cite{}.
В рамках стека мультимедиа-сетевых технологий сжатые данные
передаются по сети Интернет.

Контроль битовой скорости (англ. bit-rate control) является
важнейшим элементом большей части видеокодеров. Этот
компонент определяет количество бит, затрачиваемых на один
кадр и его качество. Существуют два типа контроля битовой
скорости выхода видеокодера: ориентирование на постоянную
(англ. constant bit-rate) и переменную (англ. variable bit-rate, VBR)
битовые скорости. При постоянной битовой скорости задачей
видеокодера является оптимизация качества передаваемого
видео при максимальном приближении размера сжатого кадра
к требуемому~\cite{}. Это приводит к колебаниям качества
при смене сцен, освещения и сценах с повышенным
движением в кадре. Видеокодеры с переменной битовой
скоростью используются при необходимости поддерживать постоянное
качество видеосигнала и при смягчённых требованиях
передачи видеотрафика в реальном времени.

В данной работе рассматривается моделирование
сжатого видеотрафика с переменной битовой скоростью.

Моделирование видеотрафика позволяет контролировать
пригодность сети для использования в качестве транспорта
при передаче видео на этапе её проектирования,
а также использовать искусственно сгенерированный трафик
при нагрузочном тестировании существующей сети.

В литературе было предложено большое количество моделей
видеотрафика, адаптированных под видеоданные разной природы.
В данной работе описаны основные подходы к классификации
и моделированию видеотрафика. Проанализированы особенности
конкретного типа видеотрафика --- трафика видеоконференций
--- для которого реализованы две модели:
дискретно-авторегрессионная и марковская. Для марковской
модели предложена и реализована оптимизация, позволяющая
уменьшить количество ``вырожденных'' состояний.

Вопрос сравнения различных моделей остаётся открытым:
модели видеотрафика часто базируются на совершенно
различных математических аппаратах и принципах.
Какие критерии позволят сделать вывод о превосходстве
одной модели над другими? В данной дипломной работе
рассмотрены основные методы сравнения моделей и
предложен дополнительный метод сравнения моделей
на основе оценки задержки и джиттера.

Осуществлён сравнительный анализ реализованных моделей
для стандартных тестовых видео базы Государственного
Университета Аризоны~\cite{}, которая используется
специалистами в области обработки мультимедиа в качестве
``общего знаменателя'', и для самостоятельно записанных
видео подходящего типа, необходимость в которых объясняется
отсутствием в базе тестовых видео требуемой длительности
в широком спектре разрешений.
