\sectioncentered*{Заключение}
\addcontentsline{toc}{section}{Заключение}

В данной дипломной работе была рассмотрена задача моделирования трафика видеоконференций,
сжатого с помощью видеокодера с переменной битовой скоростью.
Были исследованы существующие подходы к моделированию видеотрафика с переменной
битовой скоростью (раздел~\ref{sec:survey}) и выбраны подходы и конкретные модели, в наибольшей степени
подходящие для моделирования трафика видеоконференций.

Были реализованы две из существующих моделей: дискретная авторегрессионная модель
произвольного порядка DAR(p) (модель порядка более высокого, чем DAR(1), в литературе
не встречалась) и марковская модель на основе марковского возобновляемого процесса
(MRP). Для марковской модели была предложена модификация,
демонстрирующая лучшие результаты при сравнительном анализе (раздел~\ref{sec:models_description}).

Исследованы использующиеся методы сравнительного анализа моделей на основе
статистических характеристик и на основе коэффициента потерь при
переполнении буфера при передаче по сети с постоянной битовой
скоростью (раздел~\ref{sec:comparision}). Эти и предложенные автором методы были применены для анализа
адекватности моделей в имитировании видеопоследовательностей тестового
множества.

Была предложена методика сравнительного анализа моделей на основе
задержки и джиттера, предоставляющая дополнительную информацию
об адекватности модели. Использование этих характеристик
качества обслуживания уже предлагалось в литературе~\cite{survey2013}, но
на практике задержка и джиттер не использовались таким образом.

Сравнительный анализ моделей (раздел~\ref{sec:results}) показал адекватность использования
выбранных моделей для имитации трафика видеоконференций плохого
и среднего качества, однако выявил долгосрочные зависимости в видеоконференциях
хорошего качества и неспособность реализованных моделей их
отражать. Наличие долгосрочных зависимостей в трафике данного типа
не обнаружено в исследовательских работах~\cite{survey2013, ars2004, characteristics2013},
что позволяет говорить о необходимости в дополнительном исследовании
моделирования трафика HD-видеоконференций.
